%search for nnnote for things to resolve later.

\RequirePackage[l2tabu, orthodox]{nag}

\documentclass[letterpaper, 11 pt]{article}

\usepackage[T2A, T1]{fontenc}
\usepackage[utf8]{inputenc}
\usepackage{lmodern}
\usepackage{CJKutf8}
\usepackage{amsfonts}
\usepackage{amsmath}
\usepackage{amssymb}
\usepackage{ellipsis}
\usepackage{microtype}
\usepackage[ocgcolorlinks]{hyperref}
\usepackage[numbers,sort&compress]{natbib}

%change link colours:
\hypersetup{ linktocpage, colorlinks=true, linkcolor=blue,
             citecolor=blue, filecolor=blue, urlcolor=blue }

%Bibliography stuff:
\bibliographystyle{ieeetr}
\usepackage{etoolbox}
\apptocmd{\thebibliography}{\raggedright}{}{}

\title{Dissertation: early version}
\author{Matthew Baxter}
\date{\today}

\begin{document}

\maketitle

\begin{section}{Introduction \label{sec:intro}}

\end{section}

\begin{section}{Density Functional Theory \label{chap:dft}}

   \begin{subsection}{Ground state DFT \label{sec:dft}}

      In the standard treatment of many-body quantum mechanics a system is described by a wavefunction
      $\Psi$ which, depending upon the situation, is determined by either time-dependent Sch\"{o}dinger
      equation (TDSE)
      %
      \begin{equation} \label{eq:tdse}
         \hat{H}(t) \Psi(t) = i \frac{\mathrm{d} \Psi(t)}{\mathrm{d} t},
      \end{equation}
      %
      or the stationary Sch\"{o}dinger equation (SSE)
      %
      \begin{equation} \label{eq:sse}
         \hat{H} \Psi = E \Psi.
      \end{equation}
      %
      If the system in question consists of $N$ non-relativistic electrons then $\Psi$ becomes a
      function $N$ position variables $\mathbf{r}_i$ and spin variables $\sigma_i$. The Hamiltonian,
      $\hat{H}$, may be decomposed into a kinetic energy term
      %
      \begin{equation} \label{eq:Top} %nnnote: tfrac or normal frac?
         \hat{T} = -\frac{1}{2} \sum\limits^{N}_{i=1} \Delta_i,
      \end{equation}
      %
      an electron-electron term
      %
      \begin{equation} \label{eq:Vee} %nnnote: tfrac or normal frac?
         \hat{V}_{ee} = \frac{1}{2} \sum\limits^{N}_{i \neq j}
                        \frac{1}{\left| \mathbf{r}_i - \mathbf{r}_j \right|},
      \end{equation}
      %
      and a, possibly, time-dependent external potential
      %
      \begin{equation} \label{eq:Vext}
         \hat{V}_\mathrm{ext} = \sum\limits^{n}_{i = 1} v_\mathrm{ext} (\mathbf{r}_i, \sigma_i, t).
      \end{equation}

   \end{subsection}

   \begin{itemize}

      \item Introduce ground state DFT~\cite[p. 61]{dft-engel}.

      \item Hohenberg–Kohn~\cite{hk-theorem}, extension to spin-polarized
         systems~\cite{spin-dep1, spin-dep2}. KS system~\cite{ks-eq}, non-interacting
         v-representability~\cite{nonint1, nonint2}, KS with spin:~\cite{spin-dep1, spin-dep3}.

      \item $E_\mathrm{xc}[n]$ is a universal functional~\cite{dft-engel}.

      \item Discuss some details of how things are determined: LDA, GGAs, et cetera
         ({\color{red}{still need this}}).

      \item Topics of $v$-representability (lattice~\cite{vrep-lat}) (general~\cite{vrep-levy1,
         vrep-levy2, vrep-lieb}, review:~\cite{vrep-rev}). References include stuff about domain of
         definition of the functionals (where functional derivatives exist and such).

      \item TDDFT~\cite{rgt, tddft, marques-1}, extension to spin-polarized systems:~\cite{td-spindep}.

      \item Runge-Gross~\cite{rgt}.

      \item QM action:~\cite{qmaction}, causality problems:~\cite{tddft-causality},
         various solutions:~\cite{caus-sol1, caus-sol2}, action functional well
         defined:~\cite{td-welldef}, 

      \item TDDFT v-rep~\cite{td-vrep}.

      \item Various xc-potentials {\color{red}{Add some refs?}}.

      \item OPM: DTF~\cite{opm1, opm2} other derivations:~\cite{opm3, opm4, opm5, opm-rev} and
         KLI: various derivations~\cite{kli1, kli2, kli3},
         Various applications:~\cite[p. 254]{dft-engel},
         TD-OPM:~\cite{tdopm}, TD-KLI:~\cite{tdkli1, tdkli2, tdkli3}, Various problems with
         TD-KLI:~\cite[p. 134-135]{tddft}.

   \end{itemize}

\end{section}

\begin{section}{Observables \label{sec:obs}}
   
   \begin{itemize}
      
      \item Introduce observable problem: $p$-He, He\textsuperscript{2+}-He.

      \item Discuss observables of interest in the present case.

      \item IEM.

      \item WB model~\cite{wb}.

      \item expansion.

      \item Details: p-b and/or p-z plots, contour plot of density difference et cetera.

      \item Results/discussion.

   \end{itemize}

\end{section}

\begin{section}{\texorpdfstring{He\textsuperscript{+}}{He+}-He \label{sec:hep-he}}

\end{section}

\begin{section}{Conclusion \label{sec:con}}

\end{section}

\begin{section}*{Appendices}

\end{section}

\bibliography{diss.bib}

\end{document}

