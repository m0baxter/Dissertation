%search for nnnote for things to resolve later.

\RequirePackage[l2tabu, orthodox]{nag}

\documentclass[letterpaper, 11 pt]{article}

\usepackage[T2A, T1]{fontenc}
\usepackage[utf8]{inputenc}
\usepackage{lmodern}
\usepackage{CJKutf8}
\usepackage{amsfonts}
\usepackage{amsmath}
\usepackage{amssymb}
\usepackage{braket}
\usepackage{ellipsis}
\usepackage{microtype}
\usepackage[ocgcolorlinks]{hyperref}
\usepackage[numbers,sort&compress]{natbib}

%change link colours:
\hypersetup{ linktocpage, colorlinks=true, linkcolor=blue,
             citecolor=blue, filecolor=blue, urlcolor=blue }

%Bibliography stuff:
\bibliographystyle{ieeetr}
\usepackage{etoolbox}
\apptocmd{\thebibliography}{\raggedright}{}{}

\title{Dissertation: early version}
\author{Matthew Baxter}
\date{\today}

\begin{document}

\maketitle

\begin{section}{Introduction \label{sec:intro}}

\end{section}

\begin{section}{Density Functional Theory \label{chap:dft}}

   \begin{subsection}{DFT and TDDFT Existence Theorems \label{sec:dft}}

      In the standard treatment of many-body quantum mechanics a system is described by a wavefunction
      $\Psi$ which, depending upon the situation, is determined by either time-dependent Sch\"{o}dinger
      equation (TDSE)
      %
      \begin{equation} \label{eq:tdse}
         \hat{H}(t) \Psi(t) = i \frac{\mathrm{d} \Psi(t)}{\mathrm{d} t},
      \end{equation}
      %
      or the stationary Sch\"{o}dinger equation (SSE)
      %
      \begin{equation} \label{eq:sse}
         \hat{H} \Psi = E \Psi.
      \end{equation}
      %
      If the system in question consists of $N$ non-relativistic electrons then $\Psi$ becomes a
      function $N$ position variables $\mathbf{r}_i$ and spin variables $\sigma_i$. The Hamiltonian,
      $\hat{H}$, may be decomposed into a kinetic energy term
      %
      \begin{equation} \label{eq:Top} %nnnote: tfrac or normal frac?
         \hat{T} = -\frac{1}{2} \sum\limits^{N}_{i=1} \Delta_i,
      \end{equation}
      %
      an electron-electron term
      %
      \begin{equation} \label{eq:Vee} %nnnote: tfrac or normal frac?
         \hat{V}_{ee} = \frac{1}{2} \sum\limits^{N}_{i \neq j}
                        \frac{1}{\left| \mathbf{r}_i - \mathbf{r}_j \right|},
      \end{equation}
      %
      and a, possibly, time-dependent external potential
      %
      \begin{equation} \label{eq:Vext}
         \hat{V}_\mathrm{ext} = \sum\limits^{n}_{i = 1} v_\mathrm{ext} (\mathbf{r}_i, \sigma_i, t).
      \end{equation}
      %
      The function $\hat{V}_\mathrm{ext}$ contains all of the one-body interactions, including the
      nuclear and any external potentials.

      Given the one-particle electronic density
      %
      \begin{equation} \label{eq:dendef1}
         n(\mathbf{r}, t) = N \sum\limits_{\sigma_i} \int \mathrm{d}^3 r_2 \dots \mathrm{d}^3 r_N
                            \left| \Psi(\mathbf{r}, \sigma_1, \mathbf{r}_2, \sigma_2, \dots,
                                   \mathbf{r}_N, \sigma_N) \right|^2
      \end{equation}
      %
      the Hohenberg–Kohn theorem~\cite{hk-theorem} in the stationary case and the Runge-Gross
      theorem~\cite{rgt} in the time-dependent case establish a one-to-one mapping between the
      one-particle density $n$ and the external potential $\hat{V}_\mathrm{ext}$. The potential is
      then a unique functional of the one-particle density
      %
      \begin{equation} \label{eq:vext-func}
         \hat{V}_\mathrm{ext} = \hat{V}_\mathrm{ext} [n].
      \end{equation}
      %
      It should be noted that for time-dependent systems this mapping is unique only up to the addition
      of an arbitrary time-dependent function, as this serves only to introduce a phase into the
      associated wave function $\Psi[\hat{V}_\mathrm{ext}]$ it can be safely ignored in all future
      discussions.

      We have formulated the density-potential mapping for an explicitly spin-dependent system, it
      then be noted that the original existence theorems we not formulated in such general terms.
      Luckily, generalizations for both the stationary~\cite{spin-dep1, spin-dep2} and
      time-dependent~\cite{td-spindep} to spin-polarized systems exist.

   \end{subsection}

   \begin{subsection}{The Kohn-Sham Equations \label{sec:ks}}

      In practice the correspondence between the density and potential is used to map the interacting
      many-body SSE or TDSE onto an auxiliary non-interacting system. The density-potential mappings
      discussed in the previous section allow one to rewrite the interacting system in terms of
      an auxiliary system of non-interacting particles by the functions $\varphi_{i\sigma}$ ($i = 1,
      \dots, N$) with
      %
      \begin{equation} \label{eq:dendef2}
         n = \sum\limits_{\sigma} \sum\limits_{i = 1}^N
                           \left| \varphi_{i\sigma} \right|^2,
      \end{equation}
      %
      where $n$ is the one-particle density of the fully interacting system. The orbitals $\phi_i$ are
      determined through the stationary or time-dependent Kohn-Sham equations~\cite{ks-eq, spin-dep1,
      spin-dep3} (SKS, TDKS respectively)
      %
      \begin{equation} \label{eq:sks}
         \left( -\frac{\Delta}{2} + v^\sigma_\mathrm{KS}[n_\uparrow, n_\downarrow](\mathbf{r}) \right)
          \varphi_{i\sigma}(\mathbf{r}) = \epsilon_i \varphi_{i \sigma}(\mathbf{r}),
      \end{equation}
      %
      \begin{equation} \label{eq:tdks}
         i \frac{\partial}{\partial t} \varphi_{i\sigma} = 
            \left( -\frac{\Delta}{2} + v^\sigma_\mathrm{KS}[n_\uparrow, n_\downarrow](\mathbf{r},t)
            \right) \varphi_{i\sigma}(\mathbf{r},t),
      \end{equation}
      %
      where the $\epsilon_i$ appearing in Eq.\ \eqref{eq:sks} are the Kohn-Sham eigenvalues and the
      quantities $n_\uparrow$, $n_\downarrow$ are the spin-up/down one-particle densities defined by
      %
      \begin{equation} \label{eq:spinden}
         n_\sigma = \sum\limits_{i=1}^{N} \left| \phi_{i\sigma} \right|^2.
      \end{equation}

      The potential in Eqs.\ \eqref{eq:sks} and \eqref{eq:tdks} is know as the Kohn-Sham potential. This
      potential may be simplified by splitting it into a series of less complex objects
      %
      \begin{equation} \label{eq:vks}
         v^\sigma_\mathrm{KS}[n_\uparrow, n_\downarrow] = v_\mathrm{ext} + v_\mathrm{H}
            + v_\mathrm{xc}[n_\uparrow, n_\downarrow].
      \end{equation}
      %
      The first term in this expression is the external potential, which is essentially the same as the
      potential $\hat{V}_\mathrm{ext}$ of Eqs.\ \eqref{eq:sse} and \eqref{eq:tdse}. The next term is
      the Hartree screening potential
      %
      \begin{equation} \label{eq:vh}
         v_\mathrm{H}(\mathbf{r},t) = \int \frac{n(\mathbf{r}^\prime, t)}
            {\left| \mathbf{r} - \mathbf{r}^\prime\right|} \, \mathrm{d}^3 r^\prime.
      \end{equation}
      %
      The last term is the exchange-correlation potential which encodes the complicated
      electron-electron interaction potential into the language of the non-interacting system. For
      convenience this is often further broken down into separate exchange and correlation potentials
      %
      \begin{equation} \label{eq:vxc}
         v_\mathrm{xc} = v_\mathrm{x} + v_\mathrm{c}.
      \end{equation}

   \end{subsection}

   \begin{subsection}{Observables \label{sec:obs}}

      In the standard treatment of many body quantum mechanics there is a well established process for
      calculating observables from the full many body wave function describing the system. For any
      solution $\Psi$, unique up to a phase factor, of the SSE or TDSE and any observable $\hat{O}$ we
      immediately have the unique functional
      %
      \begin{equation} \label{eq:obsfunc1}
         O[\Psi] = \braket{\Psi| \hat{O} | \Psi},
      \end{equation}
      %
      so long as $\hat{O}$ contains no time derivative terms.

      The density-potential mappings of Sec.\ \ref{sec:dft} provide the relation
      %
      \begin{equation} \label{eq:denpot}
         n \mapsto \hat{V}_\mathrm{ext}[n] + c(t).
      \end{equation}
      %
      As the function $c$ only serves to introduce another phase factor we may use the uniqueness of
      solutions of the SSE/TDSE to define a map
      %
      \begin{equation} \label{eq:obsfunc2}
         n \mapsto \hat{V}_\mathrm{ext} \mapsto \Psi \mapsto O
      \end{equation}
      %
      or, $O = O[n]$; the observable is a functional of the one-particle density.

      In principle all observables are functionals of the one-particle density. However, in practice the
      exact functional is only known in a handful of cases \cite[p. 211-213]{obs_exac}.

   \end{subsection}

   \begin{itemize}

      \item Introduce ground state DFT~\cite[p. 61]{dft-engel}.

      \item non-interacting v-representability~\cite{nonint1, nonint2}.

      \item $E_\mathrm{xc}[n]$ is a universal functional~\cite{dft-engel}.

      \item Discuss some details of how things are determined: LDA, GGAs, et cetera
         ({\color{red}{still need this}}).

      \item Topics of $v$-representability (lattice~\cite{vrep-lat}) (general~\cite{vrep-levy1,
         vrep-levy2, vrep-lieb}, review:~\cite{vrep-rev}). References include stuff about domain of
         definition of the functionals (where functional derivatives exist and such).

      \item TDDFT~\cite{rgt, tddft, marques-1}.

      \item QM action:~\cite{qmaction}, causality problems:~\cite{tddft-causality},
         various solutions:~\cite{caus-sol1, caus-sol2}, action functional well
         defined:~\cite{td-welldef}, 

      \item TDDFT v-rep~\cite{td-vrep}.

      \item Various xc-potentials {\color{red}{Add some refs?}}.

      \item OPM: DTF~\cite{opm1, opm2} other derivations:~\cite{opm3, opm4, opm5, opm-rev} and
         KLI: various derivations~\cite{kli1, kli2, kli3},
         Various applications:~\cite[p. 254]{dft-engel},
         TD-OPM:~\cite{tdopm}, TD-KLI:~\cite{tdkli1, tdkli2, tdkli3}, Various problems with
         TD-KLI:~\cite[p. 134-135]{tddft}.

   \end{itemize}

\end{section}

\begin{section}{Observables \label{sec:p-he2p-he}}
   
   \begin{itemize}
      
      \item Introduce observable problem: $p$-He, He\textsuperscript{2+}-He.

      \item Discuss observables of interest in the present case.

      \item IEM.

      \item WB model~\cite{wb}.

      \item expansion.

      \item Details: p-b and/or p-z plots, contour plot of density difference et cetera.

      \item Results/discussion.

   \end{itemize}

\end{section}

\begin{section}{\texorpdfstring{He\textsuperscript{+}}{He+}-He \label{sec:hep-he}}

\end{section}

\begin{section}{Conclusion \label{sec:con}}

\end{section}

\begin{section}*{Appendices}

\end{section}

\bibliography{diss.bib}

\end{document}

